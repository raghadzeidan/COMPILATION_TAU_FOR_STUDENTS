%%%%%%%%%%%%%%%%%%%%%%%%%%%%%%%%
% SECTION :: IR in our project %
%%%%%%%%%%%%%%%%%%%%%%%%%%%%%%%%
\section{IR in our project}
%%%%%%%%%%%%%%%%%%%%%%%%%%%%%%%%%%%
% Frame Open :: IR in our project %
%%%%%%%%%%%%%%%%%%%%%%%%%%%%%%%%%%%
\frame{\frametitle{IR in our project :: Overflow handled in AST $\rightarrow$ IR}
\begin{itemize}
\item Handling arithmetic overflow in the AST $\rightarrow$ IR phase
      will yield the following IR code for the addition above:
%%%%%%%%%%%%%%%%%%
% Table :: Begin %
%%%%%%%%%%%%%%%%%%
\begin{table}[h]
\centering
\begin{tabular}{ l }
%%%%%%%%%%%%%%%%%%%%%%%%%%%%%%%%%%%%%%%%%%%
                                         \\
li Temp\_24, 32765                       \\
li Temp\_25, 8                           \\
add Temp\_23, Temp\_24, Temp\_25         \\
li Temp\_26,  32767                      \\
li Temp\_27, -32768                      \\
bgt Temp\_23, Temp\_26, Label\_Overflow  \\
blt Temp\_23, Temp\_27, Label\_Underflow \\
\# What should we write here?            \\
Label\_Overflow:                         \\
\# and here?                             \\
Label\_Underflow:                        \\
\# and here too?                         \\
Label\_End:                              \\
%%%%%%%%%%%%%%%%%%%%%%%%%%%%%%%%%%%%%%%%%%%
\end{tabular}
\caption*{\label{Table_Leaving_Overflow_To_IR_2_MIPS}}
\end{table}
\end{itemize}

%%%%%%%%%%%%%%%%%%%%%%%%%%%%%%%%%%%
% Frame Close :: Warm up examples %
%%%%%%%%%%%%%%%%%%%%%%%%%%%%%%%%%%%
}